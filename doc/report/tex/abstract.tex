\begin{abstract}

    Sztuczne sieci neuronowe na stałe zadomowiły się w~dziedzinie, którą dzisiaj powszechnie określamy mianem sztucznej inteligencji. Algorytmy tworzone przez takie firmy jak Google potrafią już same uczyć się operowania w~tak złożonych grach jak szachy czy Starcraft II znacząco przewyższając wynikami ludzi \cite{mu_zero}. Coraz częściej pojawiają się również w~bardziej egzotycznych obszarach sterując balonami stratosferycznymi \cite{baloons} czy przewidując struktury przestrzenne długich łańcuchów aminokwasowych \cite{proteins}.
    
    Jednym z~klasycznych zastosowań sieci neuronowych jest klasyfikacja obrazów. Wśród najpowszechniej używanych w~tym celu architektur znajduje się od dłuższego czasu zaproponowana w~2014 roku \textit{VGG} \cite{vgg19}. Niniejsza praca skupia się na jednym z~wariantów tego modelu -~VGG19~- analiząc jego możliwości w~kontekście klasyfikacji obrazów owoców ze zbioru \textit{Fruits-360} \cite{fruits-360}. Pierwsze trzy rozdziały stanowią opis postawionego problemu, wykorzystanej architektury oraz zbioru danych. Rozdział 4~opisuje przypadki uczenia klasyfikatorów typu perceptronowego oraz SVM bazujących na cechach generowanych przez warstwy splotowe sieci VGG19 uprzednio wytrenowanej na zbiorze \textit{ImageNet}. Następnie przedstawiony został trening części klasyfikującej (typu perceptronowego) wraz z~częścią lub wszystkimi warstawmi splotowymi. Przedostatni rozdział zgłębia analizę wytrenowanych sieci wykorzystując techniki wizualizacji obszarów uwagi oraz stopnia aktywacji poszczególnych warstw sieci.

\end{abstract}